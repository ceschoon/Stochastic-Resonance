%% SECTION: INTRODUCTION

\paragraph{} Durant les deux derniers millions d'années, des épisodes de glaciation et de réchauf- fement planétaires se sont succédés, à des intervalles étonnamment réguliers dans le temps. Depuis le début du XXe siècle est émis l'hypothèse que ce changement quasi-périodique du climat Terrestre est une conséquence de variations dans le flux de chaleur Solaire. C'est l'astronome serbe Milutin Milankovitch qui en calculant les variations de paramètres de la trajectoire de la Terre autour du Soleil, désormais appelées cycles de Milankovitch, a remarqué leur corrélation avec les variations climatiques. Cependant, la communauté scientifique est encore aujourd'hui dépourvue d'une théorie solide expliquant le phénomène. Il est difficile d'identifier les mécanismes reliant l'évolution des paramètres orbitaux à l'évolution observée du climat. 

\paragraph{} Dans ce travail, nous allons commencer par présenter le problème des glaciations ainsi que les explications qui ont été proposées sur base des cycles de Milankovitch. Dans un second temps, nous nous intéresserons au phénomène de résonance stochas-tique, découvert au début des années 1980 par C. Nicolis \cite{nicolis1981} ainsi que Benzi et al. \cite{benzi1981}. Ce mécanisme assez contre-intuitif décrit l'amplification de la réponse d'un système à un forçage périodique, par le bruit de l'environnement dans lequel le système est plongé. Comme souligné par ces découvreurs, ce phénomène semble tout à fait pertinent pour comprendre comment le forçage induit par les cycles de Milankovitch peut causer une réponse climatique. 

