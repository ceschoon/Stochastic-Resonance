%% SECTION: LIEN ENTRE LES CYCLES DE MILANKOVITCH ET LES GLACIATIONS

%% PRINCIPAUX PBLMS
% 100ky problem
% pblm transition
% pblm causalité


%% MECANISMES
% Role de l'excentricité 
%  pblm: 1. too weak,  2. main component at 400ky and we see nothing
% Resonance climatique -> rés.stoch. solve the small amplitude pblm (1)
% Role de l'inclinaison orbitale
% Accumulation de glace sur plusieurs cycles court
% Fluctuations venant du soleil
% Couplages avec CO2 et courants océaniques (collecter réfs)
% ...

%% CRITIQUE
% dernier pblm: taille de l'échantillon et stochasticité
% bcp d'effets et corrél sur petit nbre, difficile d'extraire cause-csq
% bien que bcp de pblm, il y a consensus mais on a pas le mécanisme

%% ALTERNATIVES A MILANKOVITCH ?

\paragraph{} Les corrélations entre les données sur l'évolution du climat pendant la période quaternaire et les cycles de Milankovitch durant cette même période semblent indiquer une relation de cause à effet entre ces deux phénomènes 
\footnote{Corrélation n'implique pas causalité, mais il est difficile d'imaginer que le climat puisse influencer la trajectoire Terrestre ou qu'une cause externe et des effets à la fois sur le climat et sur la trajectoire de la Terre. %Nous savons que les cycles observés dans la trajectoire de la Terre sont causés par l'influence des autres planètes majeurs du système solaire. 
Les seule possibilités restantes sont que les corrélations viennent d'un effet de l'orbite sur le climat ou qu'elles sont dues au hasard.}.
Plusieurs problèmes se confrontent cependant à la théorie de Milankovitch, principalement associés à la seconde moitié du pléistocène \cite{wiki_milankovitch_cycles} \cite{huggett}. Pour commencer, le lien entre les variations de 40\,kA que semble suivre les glaciations du début du pléistocène et le cycle de l'obliquité ne tient plus dans le dernier million d'années. Aucun modèle n'a réussi à s'imposer pour expliquer les variations de 100\,kA de cette période \cite{wiki_100ky_problem}. Une des difficultés présentée par ces oscillations réside dans la non-linéarité de la réponse au forçage, contrairement aux oscillations avant la transition. Ensuite, peu d'explications on pu être apportées pour expliquer la transition entre ces deux régimes. Un seul modèle récent a pu reproduire cette transition dans un modèle numérique tentant compte de l'évolution du taux de CO$_2$ et de l'érosion des sols par la calotte de glace \cite{willeit2019}. Enfin, la causalité du phénomène peut parfois être remise en question, comme lors du réchauffement s'étant produit il y a 135\,kA, qui précède l'augmentation du flux solaire d'environ 10\,kA \cite{karner2000}.

\paragraph{} Les périodicités de 100\,kA du dernier million d'années sont souvent attribuées au cycle de l'excentricité, du fait de la proximité avec ses deux composantes à 95 et 125\,kA. Toutefois, cela pose deux problèmes. Le premier est que ce cycle a une composante importante à 400\,kA, qui n'est pas visible dans les variations climatiques. Le second est que les variations du flux solaire causée par le cycle d'excentricité sont beaucoup plus faible que celles provoquées par les cycles d'obliquité et de précession axiale, d'environ 1\%-2\% \cite{ruddiman2006}.

\paragraph{} Un mécanisme qui peut expliquer à la fois l'influence d'une fréquence particulière du forçage et le problème de la petite amplitude présenté au paragraphe précédent réside dans la notion de résonance. Une sensibilité de la dynamique Terrestre à des fréquences de forçage particulières expliquerait comment un cycle spécifique peut régir le climat de la Terre, son effet étant amplifié par la résonance. Le mécanisme de résonance stochastique que nous approfondissons plus loin est un exemple de tel phénomène. 

\paragraph{} Il est également possible que les oscillations de 100\,kA ne sont pas dues aux cycles de l'excentricité. Des explications alternatives existent. En particulier, une hypothèse d'effet indirect a été proposé \cite{muller1995} à travers les variations de l'inclinaison orbitale. L'inclinaison orbitale ne change pas directement le flux solaire mais, en changeant sa course, la Terre pourrait passer dans des nuages de poussières qui affecteraient ensuite le climat. Cette théorie a l'avantage de ne pas présenter le problème de causalité subit par les autres cycles.

\paragraph{} D'autres explications reprennent les cycles de l'obliquité et de la précession axiale ainsi qu'à des effets amplificateurs comme par les gaz à effet de serre. Une hypothèse est que les terminaisons des glaciations de la deuxième moitié du pléistocène ne se produisent plus à chaque cycle. À la place, la glace ne fondrait plus totalement et s'accumulerait progressivement pour ensuite fondre rapidement à des multiples de la période du cycle d'obliquité, dû à une réponse amplifiée par la présence de CO$_2$ \cite{ruddiman2006}. Un tel modèle donne des réchauffement espacés d'environ 80 ou 120\,kA, ce qui semble assez bien décrire l'évolution du climat à cette période. De plus, la transition du milieu du pléistocène s'expliquerait par le changement entre un régime où la fonte de la calotte est totale à chaque cycle et un autre où la fonte est partielle.

\paragraph{} Pour finir, il a été proposé que des variations de l'intensité solaire provenant de l'activité du Soleil même pourraient être à l'origine des cycles de glaciations, via des ondes de diffusions \cite{ehrlich2007}. 

\paragraph{} Malgré toutes ces propositions, le problème posé par les dernières glaciations reste ouvert. Une difficulté importante de cette énigme est la petitesse de l'intervalle de temps en question. Seulement 900\,kA de données sont disponible pour décrire un effet oscillatoire de période 100\,kA, ce qui n'est pas suffisant pour discriminer plusieurs explications concurrentes. La proposition que le climat est contrôlé de manière déterministe par les forçages orbitaux est elle-même fortement discutable et l'on ne peut écarter l'idée que ces oscillations sont principalement le résultat d'un processus stochastique \cite{wunsch2004}.
 
\paragraph{} Au final, en dépit de ses nombreux problèmes, la théorie de Milankovitch fait un consensus assez large dans la communauté scientifique. %\cite{wiki_milankovitch_cycles} \cite{wunsch2004}